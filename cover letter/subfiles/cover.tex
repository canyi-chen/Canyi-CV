% !TEX program = pdflatex
%% !BIB program = biber
% !TeX encoding = UTF-8
% !TeX root = ../application.tex
\documentclass[../application.tex]{subfiles}


\begin{document}

\thispagestyle{empty} \vskip 5mm
\baselineskip 17pt
\renewcommand{\thepage}{}
%\include{titre}
\pagenumbering{arabic}


% --- Institution Logo ---
% skip 1in space for margins
\vspace*{-1in}
	\includegraphics[width=0.4\textwidth,left]{../figs/sph-logo.png}\par
% \begin{figure*}
% 	\includegraphics[width=0.4\textwidth,left]{../figs/sph-logo.png}
% \end{figure*}

% --- Sender Information ---
\begin{flushright}
    \textbf{Canyi Chen}\\
    Postdoctoral Research Fellow\\
    Department of Biostatistics, University of Michigan\\
    Ann Arbor, MI 48109, USA\\
    \href{mailto:canyic@umich.edu}{canyic@umich.edu} $|$ (+86) 130-5151-7009\\
    \href{https://canyi-chen.github.io}{https://canyi-chen.github.io}
\end{flushright}

\vspace{0.5cm}

% --- Date ---
\today

\vspace{0.5cm}

% --- Recipient Information ---
\noindent
\textbf{Faculty Search Committee}\\
Department of Statistics and Data Science\\
Tsinghua University\\
Haidian District, Beijing 100084, P. R. China

\vspace{1cm}

% \noindent
% {Canyi Chen} \hfill Phone: (+86) 13051517009\\
% Postdoctoral Research Fellow  \hfill E-mail: \texttt{canyic@umich.edu} \\
% Ann Arbor, MI 48109, USA \hfill \url{https://canyi-chen.github.io} \\
% [0.2cm]
% \today\\
% [0.2cm]


% \noindent
% Faculty Search Committee\\
% Department of Statistics and Data Science\\
% Tsinghua University\\
% Haidian District, Beijing 100084, P. R. China\\
% [0.2cm]

% --- Salutation ---
\noindent Dear Members of the Faculty Search Committee,

% --- Opening: The Brand & Strong Opening Statement  ---
I am writing to apply for the tenure-track Assistant Professor position in the Department of Statistics and Data Science at Tsinghua University, beginning in the 2026--2027 academic year. I am currently a Postdoctoral Research Fellow at the University of Michigan, working with Dr. Peter Song, and I earned my Ph.D. in Statistics from Renmin University of China in June 2023, supervised by Dr. Liping Zhu. My research in {\textit{distributed statistical learning}}, {\textit{causal mediation analysis}}, and {\textit{GenAI-enhanced inference}} develops statistically principled methodologies that ensure reliable inference in large-scale, heterogeneous, and complex data environments. These contributions, I believe, would be a great fit with Tsinghua's strategic vision at the confluence of statistical methodology, optimization, AI, and interdisciplinary data science.

% --- Paragraph 2: Scientific Achievements & Productivity  ---
My research agenda centers on building statistically grounded frameworks that ensure robustness, efficiency, and inferential validity for complex, large-scale data systems. To date, I have produced 30 manuscripts, including 16 peer-reviewed articles in leading statistics journals such as {\it JASA}, {\it JCGS}, and {\it Statistica Sinica}, with additional work currently under review in {\it Biometrika} and {\it Biometrics}. In {\it distributed statistical learning}, I demonstrate that robust inference, communication efficiency, and optimization scalability can be simultaneously achieved in decentralized and adversarial environments. By integrating kernel smoothing, robust estimation, and nonsmooth optimization, I have developed algorithms that maintain statistical guarantees while accommodating data heterogeneity and contamination. In {\it causal mediation analysis}, I have introduced calibrated testing procedures and dependence-aware FDR control methods that enable valid inference on mediation pathways under complex dependence structures.

% --- Paragraph 3: Motivation, Impact & Fit (The "Niche")  ---
I believe my research strongly resonates with Tsinghua University's mission to advance foundational statistics in support of national priorities in AI and data science. In particular, my emerging work on {\textit{GenAI-enhanced inference}}---developing principled stopping criteria for inference safeguards and then enhancements---would complement the department's growing strengths in the era of LLMs. My long-term vision is to advance optimization in statistics while developing the rigorous methodological toolbox needed for integrating GenAI into statistical science.

% --- Paragraph 4: Teaching & Mentorship [cite: 90] ---
I am deeply committed to teaching and mentorship. I have served as a teaching assistant for Ph.D.-level courses in data science computing and asymptotic statistics, where I emphasized both theoretical foundations and algorithmic implementation. As a research mentor, I have guided junior scholars whose projects have culminated in publications in journals such as {\it JCGS} and {\it Statistica Sinica}. I am prepared to teach undergraduate courses in probability and mathematical statistics, as well as graduate seminars on distributed learning, causal inference, and high-dimensional inference.

% --- Paragraph 5: Service & Closing [cite: 91, 94] ---
I am committed to academic service through open-source software contributions (e.g., the \texttt{SIT} and \texttt{abima} R packages) and active referee work for journals in statistics, machine learning, and data science. Enclosed are my curriculum vitae, research statement, and teaching statement. I would be honored to discuss further how I can contribute to Tsinghua University's continued excellence in statistical science.

\vspace{0.5cm}


\noindent Sincerely,\\

\begin{figure*}[!h]
	\includegraphics[width=0.15\textwidth,left]{../figs/canyi-signature.png}
\end{figure*}
%\vspace{0.3cm}
\noindent \textbf{Canyi Chen}\\
Postdoctoral Research Fellow\\
University of Michigan
% \noindent \href{mailto:canyic@umich.edu}{canyic@umich.edu}



\iffalse
I am writing to apply for the tenure-track assistant professor position in the Department of Statistics and Data Science at {\it Tsinghua University} starting in the 2026-2027 academic year. I am a Postdoc at the University of Michigan, working with Dr. Peter Song, and I completed my Ph.D. in Statistics from Renmin University of China this July under the supervision of Dr. Liping Zhu in June, 2023. My research directions in distributed statistical learning, mediation analysis, and genAI enhanced inference would complement and enrich Tsinghua's impact in the interdisciplinary of optimization, statistics, and data science. 

My research has been impactful in distributed statistial learning and mediation analysis with 32 manuscripts, including 16 peer-reviewed publications in statistics journals such as JASA, JCGS and Statistica Sinica. 




\begin{itemize}
	\item number of publication and quality
	\item advisors and wide collaborator
	\item successful mentoring experience (Nan Qiao), publication in JCGS, Sinica as corresponding author
	\item research vision: distributed, synthetic data, mediation, genAI
	\item deep and move fast in research
	\item emerging challenges from LLM and AI
\end{itemize}

I am writing to express my strong interest in the tenure-track faculty position in {\it Tsinghua University}. I am currently a Postdoctoral Research Fellow in the Department of Biostatistics at the University of Michigan, working under the supervision of Dr. Peter Song. I earned my Ph.D. in Statistics from Renmin University of China in July 2023, where I was advised by Dr. Liping Zhu.

My research focuses on distributed and large-scale statistical learning, mediation pathway analysis, and synthetic data integration—areas that address fundamental challenges in modern data science, including scalability, efficiency, and inference under complex dependence structures. During my doctoral studies, I developed distributed statistical methods for large-scale data analysis with applications in compressive sensing, robust estimation, and classification. This work has appeared in leading journals such as the \textit{Journal of the American Statistical Association} and the \textit{Journal of Computational and Graphical Statistics}. 

In my postdoctoral research, I have expanded my focus to mediation analysis, developing new testing strategies for composite null hypotheses that achieve valid type I error control and high power. These methods have been applied to study the effects of environmental exposures on health outcomes such as biological aging acceleration and childhood obesity. Currently, in collaboration with Dr. Song, I am exploring how synthetic data integration can enhance inference efficiency and false discovery control under general dependence structures. Our ongoing work in this area is under review at \textit{Biometrika} and other journals.

Looking ahead, my research vision is to develop a unified framework for distributed and synthetic-data-based inference that integrates high-dimensional modeling, causal mediation, and privacy-preserving computation. I aim to establish broad interdisciplinary collaborations in health and biomedical data science, leveraging scalable and interpretable methods to translate complex data into actionable insights.

In addition to my research, I am deeply committed to teaching and mentoring. During my graduate training, I have served as a teaching assistant for multiple statistics and data analysis courses and have mentored graduate students and interns on methodological research projects. I believe in cultivating students' statistical intuition through hands-on data exploration, critical reasoning about model assumptions, and appreciation of uncertainty quantification. I would be excited to contribute to the department's teaching mission by offering both foundational and advanced courses in statistical learning, inference, and data science.

I am particularly attracted to Florida State University's strong emphasis on methodological innovation and interdisciplinary collaboration. The department's expertise in high-dimensional statistics, statistical computing, and applied data science strongly resonates with my own research direction. I am especially interested in the possibility of collaborating with faculty members whose work intersects distributed inference, Bayesian computation, and causal modeling. I am eager to contribute to the department's growth through collaborative research, graduate mentoring, and active engagement in the academic community.

Please find my curriculum vitae and the contact information for three references attached. I would be honored to discuss how my research, teaching philosophy, and future plans align with the goals of the Department of Statistics at Florida State University. Thank you very much for your consideration, and I look forward to the opportunity to contribute to your distinguished department.\\[0.3cm]

\fi




% \vskip1.5cm
%\bibliographystyle{Chicago}
% \printbibliography%[heading=subbibliography]


\end{document}
