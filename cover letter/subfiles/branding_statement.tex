% !TEX program = pdflatex
%% !BIB program = biber
% !TeX encoding = UTF-8
% !TeX root = ../application.tex
\documentclass[../application.tex]{subfiles}



\begin{document}


% TODO:BEGIN:REFEREE3
%\iffalse
\newpage
\pagestyle{fancy}
% \setcounter{page}{1}
\resetpageaux

% --- Header ---
\begin{center}
    {\Large \textbf{Branding Statement}} \\
    % \vspace{0.2cm}
    % \textbf{Canyi Chen} \\
    % \textit{Internal Guide for Faculty Application}
\end{center}

\vspace{0.5cm}

% --- The Brand ---
\subsection*{Brand}
I am a statistical methodology researcher dedicated to advancing statistical theory, methodology, algorithms, and software for complex data structures. My work focuses on causal mediation analysis, distributed data analytics, and GenAI-enhanced inference.


% --- General Problem ---
\subsection*{General Problem}
Modern data infrastructures have evolved beyond centralized and transparent settings. Data are often geographically distributed, causal pathways are embedded in directed acyclic graphs, and an increasing portion of data is generated by opaque AI ``black-box'' systems. Classical statistical methods, originally developed for centralized, transparent, and static datasets, are therefore inadequate for delivering valid inference in these emerging environments.


% --- Specific Problem ---
\subsection*{Specific Problem}
To ensure the reliability of modern data systems, statistical methods must be capable of:
\begin{itemize}
\item Learning from \textit{distributed} and potentially adversarial data without centralized data aggregation, while preserving communication efficiency.
\item Uncovering complex \textit{causal pathways} with rigorous control of type~I error, particularly in settings where standard hypothesis tests fail to provide calibrated inference.
\item Validating and leveraging \textit{GenAI}-generated outputs (e.g., synthetic data) without compromising statistical integrity, thereby enabling safeguarded inference.
\end{itemize}



% --- Achievement ---
\subsection*{Achievement}
I have developed an integrated set of statistical frameworks that address these challenges, resulting in \textit{over 30 manuscripts}, including 16 peer-reviewed articles published in \textit{JASA}, \textit{JCGS}, and \textit{Statistica Sinica}. My research establishes that \textit{oracle-efficient inference} is attainable in distributed and adversarial environments, \textit{calibrated type~I error control} can be achieved for mediator discovery under composite null hypotheses, and \textit{optimal stopping criteria} for GenAI training can be rigorously characterized to enhance statistical power when incorporating synthetic data. These advances bridge the longstanding gap between theoretical rigor and modern computational practice.


% --- Vision ---
\subsection*{Vision}
My long-term vision is to establish a \textit{unified statistical foundation for trustworthy data science}. I aim to develop optimality theory for privacy-constrained distributed learning and to design rigorous \textit{statistical safeguards} that enable the safe integration of Generative AI into high-stakes scientific and decision-making pipelines.


\iffalse
%\centerline{\bf Point-by-Point Responses to Referee Two}
%\subsection*{Point-by-Point Responses to Referee Two}
\subsection*{{\Large\bf Branding Statement}
% \\\vskip0.2cm
% Canyi Chen
}
%\vspace{1cm}

\renewcommand{\thesubsection}{\Alph{subsection}}

% \subsection{x}



\indent {\bf Band:} I am the distributed learning, mediation analytics, and genAI-enhanced inference person. I develop statistical theory and algorithms that make large, decentralized, and AI-augmented data systems reliable for scientific discovery and decision-making.

{\bf General problem:} Modern data infrastructures are distributed, heterogeneous, and often privacy- or communication-constrained, so classical centralized procedures are no longer adequate. At the same time, many scientific and policy questions require causal pathway understanding, not just prediction, yet existing mediation methods struggle with high-dimensional mediators, complex dependence, and non-Gaussian noise. On top of this, the rapid adoption of large language models and other genAI systems raises new questions about validity, calibration, and stability when AI is embedded in data-analytic workflows.

{\bf Specific problem:}  To address these pressures, we need methods that simultaneously:
\begin{enumerate}
    \item Learn from distributed and potentially adversarial data in a communication-efficient and statistically robust way;
    \item Extract and test mediation pathways under complex dependence structures, composite nulls, and streaming or weak-signal regimes;
    \item Integrate genAI and synthetic data into inference using principled stopping rules, safeguards, and uncertainty quantification.
\end{enumerate}
My work tackles these needs through locally smoothed nonsmooth optimization for distributed learning, copula-based and quantile-based mediation frameworks with high-sensitivity joint significance tests, and genAI-enhanced statistical pipelines that use synthetic data and theory-driven stopping rules while maintaining valid type I error control.

{\bf Achivement:} I have produced 32 manuscripts, including 16 peer-reviewed publications in leading statistics journals such as JASA, JCGS, and Statistica Sinica, with additional work under review in Biometrika and Biometrics. Across these papers, I show that it is possible to achieve communication efficiency, robustness, and calibrated inference in distributed and mediation settings, and to incorporate genAI without sacrificing rigor. I have also successfully mentored junior researchers whose work with me led to publications in JCGS and Statistica Sinica, demonstrating my ability to lead coherent, high-impact research directions.

{\bf Vision:} My long-term goal is to build a unified statistical framework for trustworthy distributed learning, causal mediation, and genAI-enhanced inference. I aim to develop methods where optimization, dependence modeling, distributed computation, and LLM-driven synthesis are designed together, so that inference remains reliable and interpretable in large-scale, AI-rich environments. This agenda aligns directly with Tsinghua’s mission at the intersection of optimization, statistics, and artificial intelligence.
\fi

% \clearpage
% \vskip1.5cm
%\bibliographystyle{Chicago}
% \printbibliography%[heading=subbibliography]
% TODO:END:REFEREE3
%\fi

\end{document}
