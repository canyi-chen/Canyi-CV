% \usepackage[left=30mm,right=28mm,top=2cm,bottom=2cm]{geometry}
% \usepackage[top=0.9in,bottom=0.9in,left=0.9in,right=0.9in]{geometry}
% Margins standard for research statements
\usepackage{geometry}
\geometry{left=1in, right=1in, top=1in, bottom=1in}
\usepackage{amssymb,%amsthm,
amsmath,mathrsfs,
bm,color}
\usepackage{lineno}
\usepackage{graphicx}
\usepackage{multirow}
\usepackage{wrapfig}  % To wrap text around figures
% \usepackage{booktabs}
\usepackage{epsfig}
%\usepackage{natbib}
 \usepackage{setspace} % this will affect the length
% \usepackage{enumerate}
\usepackage[shortlabels]{enumitem}
\usepackage{threeparttable}
\usepackage{fontawesome} % For icons if desired, requires fontawesome package

\usepackage[final]{microtype}
% \setlength{\topmargin}{-0.25in} \setlength{\oddsidemargin}{0.25in}
%\setlength{\evensidemargin}{0.25in}
\usepackage{url}
%\usepackage{doi}


% \baselineskip = 7mm
% \parskip = 2.5mm

% \textheight= 9in
% \textwidth = 6.1in
% \evensidemargin = 0in
\usepackage{algorithm,algorithmic}

\usepackage{titlesec} % for redefine section
% Formatting sections
\titleformat{\section}{\large\bfseries}{\thesection}{1em}{}
\titleformat{\subsection}{\normalsize\bfseries}{\thesubsection}{1em}{}
% % section
% \titleformat{\section} % command
% {\vspace{-0.4 cm}%
% \centering\large\bfseries} % shape
% {\thesection.} % sep
% {0.5 em} % before
% {\vspace{-0.15 cm}} % after

% % subsection
% \titleformat{\subsection}
% {\vspace{-0.4 cm}%
% \centering\it%
% }
% {\thesubsection.} %sep
% {0.5em}{\vspace{-0.25 cm}}


% \newcommand{\csection}[1]
% {\begin{center}
% 		\stepcounter{section}
% 		{\bf\large\arabic{section}. #1}
% 	\end{center}
% 	\vspace{-0.15 cm}
% }

% \newcommand{\scsection}[1]
% {\begin{center}
% 		{\bf\large #1}
% 	\end{center}
% 	\vspace{-0.15 cm}
% }

% \newcommand{\csubsection}[1]{
% 	\vspace{-0.25 cm}
% 	\begin{center}
% 		\stepcounter{subsection}
% 		{\it\arabic{section}.\arabic{subsection}. #1}
% 	\end{center}
% 	\vspace{-0.25 cm}
% }

% \newcommand{\scsubsection}[1]{
% 	\vspace{-0.25 cm}
% 	\begin{center}
% 		\stepcounter{subsection}
% 		{\it #1}
% 	\end{center}

 
\numberwithin{equation}{section}



% \bibliographystyle{cheng}
%\renewcommand\refname{REFERENCE}
% \setcounter{page}{1}



\newcommand{\defn}{\stackrel{\mbox{{\tiny def}}}{=}}
\newcommand{\trans}{^{\mbox{\tiny{T}}}}
\newtheorem{theo}{\bf Theorem}%[section]
\newtheorem{lemm}{\bf Lemma}
%\newtheorem{example}{\bf Example}
\newtheorem{prop}{\bf Proposition}
\newtheorem{coro}{\bf Corollary}
% \newtheorem{exa}{\bf Example}


\def\I{{\bf I}}
\def\s{{\bf s}}
\def\b{{\bf b}}
\def\t{{\bf t}}
\def\x{{\bf x}}
\def\y{{\bf y}}
\def\m{{\bf m}}
\def\z{{\bf z}}
\def\calA{\mathcal{A}}
\def\calD{\mathcal{D}}
\def\calN{\mathcal{N}}
\def\calS{\mathcal{S}}
\def\wh{\widehat}
\def\wt{\widetilde}
\def\cov{\textrm{cov}}
% \def\var{\textrm{var}}
% \def\pr{\textrm{pr}}
\def\sp{\textrm{span}}
\def\tr{\mathrm{tr}}
\def\dist{\textrm{dist}}
\def\vecl{\textrm{vecl}}
\def\beqr{\begin{eqnarray}}
\def\eeqr{\end{eqnarray}}
\def\beqrs{\begin{eqnarray*}}
\def\eeqrs{\end{eqnarray*}}
\def\bb{\mbox{\boldmath$\beta$}}
\def\ba{\mbox{\boldmath$\alpha$}}
%\def\be{\mbox{\boldmath$\eta$}}
\def\bGam{\mbox{\boldmath$\Gamma$}}
\def\blam{\mbox{\boldmath$\lambda$}}
\def\bLam{\mbox{\boldmath$\Lambda$}}
\def\bSig{\mbox{\boldmath$\Sigma$}}
\def\beps{\mbox{\boldmath$\varepsilon$}}
\def \hDash{\bot\!\!\!\bot}
\def\mR{\mathbb{R}}
\newcommand{\mG}{\mathbb{G}}
\newcommand{\mP}{\mathbb{P}}
% \renewcommand{\baselinestretch}{1.75}

%------------In this paper----------------------------------
%------------For Algorithm--------------------------------
\renewcommand{\algorithmicrequire}{\textbf{Input:}} % Use Input in the format of Algorithm
\renewcommand{\algorithmicensure}{\textbf{Output:}} % Use Output in the format of Algorithm
\newcommand{\M}{\mathbf{M}}
\newcommand{\bS}{\mathbf{S}}
\newcommand{\X}{\mathbf{X}}
\newcommand{\rl}{^{(l)}}
\newcommand{\iid}{\emph{i.i.d.}}
\newcommand{\diag}{\mathrm{diag}}
\newcommand{\trace}{\mathrm{trace}}
%\newcommand{\lE}{\mathbb{E}}
\newcommand{\lE}{{\rm{E}}}
\newcommand{\strans}{^{*\mbox{\tiny{T}}}}
\newcommand{\Col}{\mathrm{Col}}
\newcommand{\bP}{\mathbf{P}}
\newcommand{\bOm}{\mathbf{\Omega}}
\newcommand{\cU}{\mathcal{U}}
\newcommand{\bTh}{\mathbf{\Theta}}
\newcommand{\btheta}{\bm{\theta}}
\newcommand{\bPhi}{\mathbf{\Phi}}
\newcommand{\bXi}{\mathbf{\Xi}}
\newcommand{\bT}{\mathbf{T}}
\newcommand{\bG}{\mathbf{G}}
\newcommand{\SIR}{\mathrm{SIR}}
\newcommand{\CUME}{\mathrm{CUME}}
\newcommand{\Y}{\mathbf{Y}}
\newcommand{\Yt}{\widetilde{\Y}}
\newcommand{\yt}{\widetilde{\y}}
\newcommand{\ol}{\overline}
\newcommand{\rmI}{\mathrm{I}}
\newcommand{\rmd}{\mathrm{d}}
\newcommand{\bfa}{\mathbf{a}}
\newcommand{\be}{\mathbf{e}}
\newcommand{\tred}{\textcolor{red}}
\newcommand{\rmv}{{\rm{vec}}}
\newcommand{\bE}{\mathbf{E}}
\newcommand\independent{\protect\mathpalette{\protect\independenT}{\perp}}
\def\independenT#1#2{\mathrel{\rlap{$#1#2$}\mkern2mu{#1#2}}}
\def \O{O}
\def \o{o}
\newcommand{\B}{\mathbf{B}}
\newcommand{\bC}{\mathbf{C}}
\newcommand{\V}{\mathbf{V}}
\newcommand{\U}{\mathbf{U}}
\newcommand{\D}{\mathbf{D}}
\newcommand{\bv}{\mathbf{v}}
\newcommand{\bu}{\mathbf{u}}
\newcommand{\A}{\mathbf{A}}
\def\n{\nonumber}


%------------In this paper----------------------------------
\usepackage{url}
%% code from mathabx.sty and mathabx.dcl
%% for \widecheck
\DeclareFontFamily{U}{mathx}{\hyphenchar\font45}
\DeclareFontShape{U}{mathx}{m}{n}{
	<5> <6> <7> <8> <9> <10>
	<10.95> <12> <14.4> <17.28> <20.74> <24.88>
	mathx10
}{}
\DeclareSymbolFont{mathx}{U}{mathx}{m}{n}
\DeclareFontSubstitution{U}{mathx}{m}{n}
\DeclareMathAccent{\widecheck}{0}{mathx}{"71}
\DeclareMathAccent{\wideparen}{0}{mathx}{"75}

\def\cs#1{\texttt{\char`\\#1}}
% - widecheck

% Define

\newcommand{\cl}{\mathcal{l}}
\newcommand{\cL}{\mathcal{L}}
\newcommand{\tcL}{\widetilde{\cL}}
\def\sign{\textrm{sign}}
% \newcommand{\bbeta}{\bm{\beta}}
\newcommand{\bbeta}{\bm{\zeta}}
\newcommand{\bvbeta}{\widecheck{\bbeta}}
\newcommand{\bhbeta}{\widehat{\bbeta}}
\newcommand{\btbeta}{\widetilde{\bbeta}}
\newcommand{\bhSig}{\widehat{\bSig}}
\newcommand{\bvSig}{\widecheck{\bSig}}
\newcommand{\hB}{\widehat{\B}}
\newcommand{\cH}{\mathcal{H}}
\newcommand{\argmin}{\mathop{\rm arg\min}}
\newcommand{\argmax}{\mathop{\rm arg\max}}
\newcommand{\supp}{\mathrm{supp}}
\def\Z{{\bf Z}}
\def\vz{{\check{\z}}}
\def\hz{{\hat{\z}}}
\newcommand{\cC}{\mathcal{C}}
\newcommand{\op}{\mathrm{op}}
\newcommand{\pool}{\mathrm{pool}}
%\newtheorem{remark}{\bf Remark}
% norm
\newcommand{\norm}[1]{\Vert#1\Vert}
\newcommand{\Norm}[1]{\left\Vert#1\right\Vert}
\newcommand{\inner}[2]{\langle #1, #2 \rangle}
\newcommand{\Inner}[2]{\left\langle #1, #2 \right\rangle}
\newcommand{\abs}[1]{\vert#1\vert}
\newcommand{\Abs}[1]{\left\vert#1\right\vert}


\newcommand*\from{\colon}
\newcommand{\svt}{\mathcal{T}}
\newcommand{\w}{\mathbf{w}}
\newcommand{\W}{\mathbf{W}}
\newcommand{\Bt}{\B_\ast}
\newcommand{\rank}{\textrm{rank}}

% OnlineEE
%\newcommand{\E}{\mathbb{E}}
\def\pr{\mathrm{P}}
\newcommand{\bzeros}{\bm{0}}
\newcommand{\bbetaT}{{\bbeta^\ast}}
\newcommand{\bJ}{\mathbf{J}}
%\def\var{\mathrm{Var}}
\newcommand{\Ds}{\D^\ast}
\newcommand{\bshbeta}{\bhbeta^\ast}
\newcommand{\Xa}{\mathbb{X}}

% DistributedCompositeQR
\newcommand{\calC}{\mathcal{C}}
\newcommand{\bdelta}{\bm{\delta}}
\newcommand{\bDelta}{\bm{\Delta}}
\newcommand{\balpha}{\bm{\alpha}}
\newcommand{\balphaT}{\bm{\alpha}^\ast}
\newcommand{\bhalpha}{\widehat{\balpha}}

\usepackage{caption}
\newcommand{\bmu}{\bm{\mu}}
\newcommand{\bkappa}{\bm{\kappa}}
\newcommand{\btau}{\bm{\tau}}
\newcommand{\calB}{\mathcal{B}}
\newcommand{\br}{\bm{r}}



\usepackage{tikz}
% \definecolor{offwhite}{HTML}{F2EDED}
\tikzset{> = stealth,
    hidden/.style = {
        draw = black,
        shape = circle,
        inner sep = 1pt
    }
}

\usepackage[all,import]{xy}

% \def\spacingset#1{\renewcommand{\baselinestretch}%
% 		{#1}\small\normalsize} 

\newcommand{\QNDE}{\mathrm{qNDE}}

\newcommand{\QNIE}{\mathrm{qNIE}}
\newcommand{\QTE}{\mathrm{qTE}}


%assumptions
\usepackage{hyperref}
% \hypersetup{
%     colorlinks = true,
%     linkcolor = blue
%     }
\usepackage{cleveref}

\crefname{theo}{Theorem}{Theorems}
\crefname{lemm}{Lemma}{Lemmas}
\crefname{coro}{Corollary}{Corollaries}
\crefname{prop}{Proposition}{Propositions}


% \newlist{assumptions}{enumerate}{10}
% \setlist[assumptions]{label*=(A\arabic*)}
% \crefname{assumptionsi}{assumption}{assumptions}
% \Crefname{assumptionsi}{Assumption}{Assumptions}
% \crefrangeformat{assumptionsi}{assumptions~#3#1#4--#5#2#6}
% \crefmultiformat{assumptionsi}%
% {assumptions~#2#1#3}{ and~#2#1#3}{, #2#1#3}{ and~#2#1#3}
% \crefrangemultiformat{assumptionsi}%
% {assumptions~#3#1#4--#5#2#6}{ and~#3#1#4--#5#2#6)}{, #3#1#4--#5#2#6}{ and~#3#1#4--#5#2#6}

% \Crefrangeformat{assumptionsi}{Assumptions~#3#1#4--#5#2#6}
% \Crefmultiformat{assumptionsi}%
% {Assumptions~#2#1#3}{ and~#2#1#3}{, #2#1#3}{ and~#2#1#3}
% \Crefrangemultiformat{assumptionsi}%
% {Assumptions~#3#1#4--#5#2#6}{ and~#3#1#4--#5#2#6)}{, #3#1#4--#5#2#6}{ and~#3#1#4--#5#2#6}

\Crefname{condition}{Condition}{Conditions}

\newlist{conditions}{enumerate}{10}
\setlist[conditions]{label*=(C\arabic*)}
\crefname{conditionsi}{condition}{conditions}
\Crefname{conditionsi}{Condition}{Conditions}
\crefrangeformat{conditionsi}{conditions~#3#1#4--#5#2#6}
\crefmultiformat{conditionsi}%
{conditions~#2#1#3}{ and~#2#1#3}{, #2#1#3}{ and~#2#1#3}
\crefrangemultiformat{conditionsi}%
{conditions~#3#1#4--#5#2#6}{ and~#3#1#4--#5#2#6)}{, #3#1#4--#5#2#6}{ and~#3#1#4--#5#2#6}

\Crefrangeformat{conditionsi}{Conditions~#3#1#4--#5#2#6}
\Crefmultiformat{conditionsi}%
{Conditions~#2#1#3}{ and~#2#1#3}{, #2#1#3}{ and~#2#1#3}
\Crefrangemultiformat{conditionsi}%
{Conditions~#3#1#4--#5#2#6}{ and~#3#1#4--#5#2#6)}{, #3#1#4--#5#2#6}{ and~#3#1#4--#5#2#6}

\newcommand{\calL}{\mathcal{L}}

% \newcommand{\E}{E}
\newcommand{\E}{\mathrm{E}}
\newcommand{\bF}{\mathbf{F}}
\newcommand{\bV}{\mathbf{V}}
\newcommand{\bg}{\mathbf{g}}
\newcommand{\var}{\textrm{var}}

\newcommand{\todistribution}{\xrightarrow{\text{d}}}
\newcommand{\toprobability}{\xrightarrow{\text{P}}}
\newcommand{\toas}{\xrightarrow{\text{a.s.}}}

\newcommand{\boottodistribution}{\stackrel{\text{d}^\ast}{\rightsquigarrow}}
\newcommand{\boottoprobability}{\stackrel{\text{P}^\ast}{\rightsquigarrow}}



\newcommand{\bzero}{\bm{0}}
\newcommand{\mZ}{\mathbb{Z}}
\newcommand{\bPsi}{\bm{\Psi}}



% \usepackage[toc,page,header]{appendix}
% \usepackage{minitoc}

% % Make the "Part I" text invisible
% \renewcommand \thepart{}
% \renewcommand \partname{}



\newcommand{\LT}{\mathrm{LT}}
\newcommand{\calT}{\mathcal{T}}

 % \usepackage{adjustbox}
\newcommand{\AB}{\text{ABoot}}